% Metódy inžinierskej práce

\documentclass[10pt,twoside,slovak,a4paper]{article}

\usepackage[slovak]{babel}
%\usepackage[T1]{fontenc}
\usepackage[IL2]{fontenc} % lepšia sadzba písmena Ľ než v T1
\usepackage[utf8]{inputenc}
\usepackage{graphicx}
\usepackage{url} % príkaz \url na formátovanie URL
\usepackage{hyperref} % odkazy v texte budú aktívne (pri niektorých triedach dokumentov spôsobuje posun textu)

\usepackage{cite}
%\usepackage{times}

\pagestyle{headings}

\title{Privacy Protection using Various Algorithms in Personalized Web Search\thanks{Semestrálny projekt v predmete Metódy inžinierskej práce, ak. rok 2023/24, vedenie: Vladimír Mlynarovič}} % meno a priezvisko vyučujúceho na cvičeniach

\author{Milan Šeliga\\[2pt]
	{\small Slovenská technická univerzita v Bratislave}\\
	{\small Fakulta informatiky a informačných technológií}\\
	{\small \texttt{xseligam@stuba.sk}}
	}

\date{\small 5.november 2023} % upravte



\begin{document}

\maketitle

\section{Introduction}

In an era characterized by the exponential growth of the digital landscape, our lives are increasingly intertwined with the online realm.The proliferation of smartphones, social media platforms, and smart devices has led to an  unprecedented level of connectivity, convenience, and   information access.However, this rapid expansion comes at a price - our privacy. With every digital interaction, we generate a trail of data that can be harnessed, often without our explicit consent, for various purposes. As the internet becomes an integral part of our daily lives, the need to safeguard our privacy while  conducting personalized web searches has never been more critical. This article explores the innovative algorithms designed to protect privacy in the context of personalized web search, offering insights into how individuals can regain control over their online experiences. 	

\section{Refferences}


\url{https://ieeexplore.ieee.org/document/6329891#citations}
\vspace{12pt}

%\acknowledgement{Ak niekomu chcete poďakovať\ldots}


% týmto sa generuje zoznam literatúry z obsahu súboru literatura.bib podľa toho, na čo sa v článku odkazujete
\bibliography{literatura}
\bibliographystyle{alpha} % prípadne alpha, abbrv alebo hociktorý iný
\end{document}
